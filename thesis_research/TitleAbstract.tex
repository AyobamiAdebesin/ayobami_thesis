\pagestyle{empty}
\begin{center}
%\vspace*{.1in}
\rep{The Spectral Transformation Lanczos Algorithm for the Symmetric-Definite Generalized Eigenvalue Problem: A Comparative Analysis with Conditioning Insights}{Spectral Transformation Lanczos Algorithm for Symmetric-Definite Generalized Eigenvalue Problems: A Comparative Analysis with Conditioning Insights.}

\vspace{.9in}
by\\
\vspace{.9in}
Ayobami Adebesin\\
\vspace{.9in}
Under the Direction of Michael Stewart, Ph.D. \\
\vspace{2in}

A Thesis Submitted in Partial Fulfillment of the Requirements for the Degree of\\  % Choose either thesis or dissertation and delete the other.
\vspace{.2in}
Master of Science \\
\vspace{.2in}
in the College of Arts and Sciences \\
\vspace{.2in}
Georgia State University \\
\vspace{.2in}
2025
\pagebreak 



ABSTRACT\\
\bigskip
\end{center}

\begin{flushleft}
	\justify
	This thesis investigates the application of the spectral transformation Lanczos (ST-Lanczos) algorithm to a dense symmetric-definite generalized eigenvalue problem involving real, symmetric matrices $A$ and $B$, with $B$ being positive definite and possibly\del{,} ill-conditioned. The Lanczos algorithm is a well-known iterative algorithm for computing the eigenvalues of a symmetric matrix and it works well \rep{finding extreme points in the spectrum.}{if the spectrum of the eigenvalues are well-spaced.} By leveraging a shifted and inverted formulation of the problem, the ST-Lanczos algorithm relies on iterative projection to approximate extremal eigenvalues near a shift $\sigma$.   While previous work has been done in using ST-Lanczos for sparse problems, we adapt this technique to dense problems and analyze how the error bounds already proven for direct methods plays out in an iterative context. \comm{Mostly we are testing on dense problems.  I would not say we are adapting it to dense matrices.  We simply happen to use some dense test problems because it is convenient.} \\[10pt]
	This study primarily focuses on benchmarking the ST-Lanczos method against established direct methods in the literature and addresses challenges in numerical stability, computational efficiency, and sensitivity of residuals to ill-conditioning.
\end{flushleft} 
\begin{singlespace}
\vspace{0.5in}
\noindent INDEX WORDS:
\hspace{0.2in}
\parbox[t]{4.5in}{
  eigenvalues, eigenvectors, \rep{Lanczos}{lanczos} algorithm, \rep{Ritz}{ritz} values, \rep{Krylov}{krylov} subspaces, spectral transformation, orthogonality}
\end{singlespace} 
